\documentclass[,table,dvipsnames]{article}
\usepackage[usenames, dvipsnames]{color}
\usepackage{graphicx}
\usepackage{caption}
\usepackage{subcaption}
\usepackage{hyperref}
\usepackage{longtable}
\usepackage{float}
\usepackage{tikz}
\usetikzlibrary{shapes.geometric, arrows}
\usepackage{xcolor}

\graphicspath{ {images/} {images/mat60_1AG/} }
\title{Sparse Voxel Net}
\author{xyz}
\date{Oct 2018}


\begin{document}
	
\noindent
\begin{titlepage}
	\maketitle
\end{titlepage}	

\section{Training 70 Files, 10k num point, 8A net}
\noindent\begin{tabular}{|p{10cm}|p{5.5cm}| }	
\hline \hline
\rowcolor{green!20}
RC8A  & 130 0.9633 \\
P8A & 150 0.934416532516 \\
RC8A,  edgev norm: l0 & 150 0.972146570683 \\
RC8A, No Batch norm  & 150 0.865 \\
RC8A, No Batch norm,  edgev norm: l0  & 145:0.8601 \\
 \hline 	
P8A, min0  & 150 0.9327 \\
RC8A, min0  & 150 0.9700 \\
RC8A, min0, normedgev:l0  & 150 0.96844 \\
RC8A, min0, Nbn & 145 0.8856 \\
RC8A, min0, normedgev:l0, Nbn  & 145 0.87868 \\
 \hline
RC8A, max1  & 150 0.969998 \\
 \hline 
RC8B, min0  & 150 0.9704 \\
 \hline 
\end{tabular}

\section{Training 70 Files, 15k num point, 7A net}
\noindent\begin{tabular}{|p{10cm}|p{5.5cm}| }	
	\hline 
	\rowcolor{green!20}
	RC7A  & 150, 0.9605 \\
	RC7A, min0 & 150, 0.96292 \\
	RC7A, max1 & 150, 0.93156 \\
	RC7B(no gloabl), min0 & 	150, 0.9585 \\
	RC7B, max1 & 	150, 0.9386 \\
	\hline
	P7A, min0 & 150, 0.95016\\
	P7B, min0 & 150, 0.95533\\
	\hline 
	RC7B, min0, l0 & 	150, 0.96344 \\
	RC7B, max1, l0 & 	150, 0.93648 \\
	RC7A, min0, l0 & 150, 0.95619 \\
	\hline 
	(1) Global seems helps a bit. Only a bit, some times hurt.  \par 
	(2) min0 is best, raw is almost same. max1 is bad. \par 
	(3) l0 almost no effect  & \\
	\hline
\end{tabular}
 
\end{document}